% Options for packages loaded elsewhere
\PassOptionsToPackage{unicode}{hyperref}
\PassOptionsToPackage{hyphens}{url}
\documentclass[
]{article}
\usepackage{xcolor}
\usepackage{amsmath,amssymb}
\setcounter{secnumdepth}{-\maxdimen} % remove section numbering
\usepackage{iftex}
\ifPDFTeX
  \usepackage[T1]{fontenc}
  \usepackage[utf8]{inputenc}
  \usepackage{textcomp} % provide euro and other symbols
\else % if luatex or xetex
  \usepackage{unicode-math} % this also loads fontspec
  \defaultfontfeatures{Scale=MatchLowercase}
  \defaultfontfeatures[\rmfamily]{Ligatures=TeX,Scale=1}
\fi
\usepackage{lmodern}
\ifPDFTeX\else
  % xetex/luatex font selection
\fi
% Use upquote if available, for straight quotes in verbatim environments
\IfFileExists{upquote.sty}{\usepackage{upquote}}{}
\IfFileExists{microtype.sty}{% use microtype if available
  \usepackage[]{microtype}
  \UseMicrotypeSet[protrusion]{basicmath} % disable protrusion for tt fonts
}{}
\makeatletter
\@ifundefined{KOMAClassName}{% if non-KOMA class
  \IfFileExists{parskip.sty}{%
    \usepackage{parskip}
  }{% else
    \setlength{\parindent}{0pt}
    \setlength{\parskip}{6pt plus 2pt minus 1pt}}
}{% if KOMA class
  \KOMAoptions{parskip=half}}
\makeatother
\setlength{\emergencystretch}{3em} % prevent overfull lines
\providecommand{\tightlist}{%
  \setlength{\itemsep}{0pt}\setlength{\parskip}{0pt}}
\usepackage{bookmark}
\IfFileExists{xurl.sty}{\usepackage{xurl}}{} % add URL line breaks if available
\urlstyle{same}
\hypersetup{
  hidelinks,
  pdfcreator={LaTeX via pandoc}}

\author{}
\date{}

\begin{document}

\section*{Date}\label{date}
\addcontentsline{toc}{section}{Date}

February 24(Mon)-\/-26(Wed), 2025

\section*{Venue}\label{venue}
\addcontentsline{toc}{section}{Venue}

Nagoya University, Graduate School of Mathematics

Building A, Room A207 See the map below:

https://www.math.nagoya-u.ac.jp/en/direction/nagoya.html\#b

\section*{Program}\label{program}
\addcontentsline{toc}{section}{Program}

\subsection*{2/24}\label{section}
\addcontentsline{toc}{subsection}{2/24}

\begin{description}
\item[13:00-14:30]
Stella, Cluster scattering diagrams of affine type (1)
\item[15:00-16:30]
Lee, Scattering diagrams, tight gradings, and generalized positivity (1)
\end{description}

\subsection*{2/25}\label{section-1}
\addcontentsline{toc}{subsection}{2/25}

\begin{description}
\item[9:00-10:30]
Stella, Cluster scattering diagrams of affine type (2)
\item[10:45-11:45]
Martello, Painlevé VI, symmetries, and clusters
\item[13:30-15:00]
Lee, Scattering diagrams, tight gradings, and generalized positivity (2)
\item[15:15-16:15]
Karuo, Azumaya representations of generalized skein algebras
\end{description}

\subsection*{2/26}\label{section-2}
\addcontentsline{toc}{subsection}{2/26}

\begin{description}
\item[10:00-11:30]
Stella, Cluster scattering diagrams of affine type (3)
\item[13:00-14:30]
Lee, Scattering diagrams, tight gradings, and generalized positivity (3)
\item[14:45-15:45]
Chen, A cluster theory approach from mutation invariants to Diophantine
equations
\end{description}

\section*{Abstract}\label{abstract}
\addcontentsline{toc}{section}{Abstract}

\subsection*{Salvatore Stella (Universita Degli Studi
Della'Aquila)}\label{salvatore-stella-universita-degli-studi-dellaaquila}
\addcontentsline{toc}{subsection}{Salvatore Stella (Universita Degli
Studi Della'Aquila)}

\subsubsection*{Cluster scattering diagrams of affine
type}\label{cluster-scattering-diagrams-of-affine-type}
\addcontentsline{toc}{subsubsection}{Cluster scattering diagrams of
affine type}

Cluster scattering diagrams, since their introduction, played a central
role in shaping the structure theory of cluster algebras. They consist
of a combinatorial datum, a fan, together with the assignment of a
formal power series for each of its codimension-1 cones. While their
recursive definition is in theory explicit, constructing cluster
scattering diagrams is usually a difficult task. In this lecture series
we will address this problem in the special case of acyclic cluster
algebras of affine type where we can leverage previous constructions
built using the machinery of root systems, Coxeter groups, lattice
theory to relate cluster scattering diagrams to two other fans: the
mutation fan and the fan of almost-positive roots.

\subsection*{Kyungyong Lee (The University of
Alabama)}\label{kyungyong-lee-the-university-of-alabama}
\addcontentsline{toc}{subsection}{Kyungyong Lee (The University of
Alabama)}

\subsubsection*{Scattering diagrams, tight gradings, and generalized
positivity}\label{scattering-diagrams-tight-gradings-and-generalized-positivity}
\addcontentsline{toc}{subsubsection}{Scattering diagrams, tight
gradings, and generalized positivity}

These lectures will be based on joint work with Amanda Burcroff and Lang
Mou. I plan to explain a recent result showing that the coefficients of
the wall-functions on a generalized cluster scattering diagram~of any
rank are positive, which implies the Laurent positivity for generalized
cluster algebras and the strong positivity of their theta bases.

Several known arguments allow us to reduce the general case to the rank
2 case. We will introduce generalized~rank 2 cluster algebras~and their
greedy bases/theta bases. A number of formulas for these bases will be
presented. Every statement will~be accompanied by very explicit
examples.

After we discuss bases for generalized~rank 2 cluster algebras, we will
introduce a new class of combinatorial objects which we call tight
gradings. Using this, we give a directly computable, manifestly
positive, and elementary but highly nontrivial formula describing rank 2
consistent scattering diagrams. In trying to make my lectures completely
elementary, pictorial descriptions will be given whenever possible.

\subsection*{Zhichao Chen (University of Science and Technology of
China/Nagoya
University)}\label{zhichao-chen-university-of-science-and-technology-of-chinanagoya-university}
\addcontentsline{toc}{subsection}{Zhichao Chen (University of Science
and Technology of China/Nagoya University)}

\subsubsection*{A cluster theory approach from mutation invariants to
Diophantine
equations}\label{a-cluster-theory-approach-from-mutation-invariants-to-diophantine-equations}
\addcontentsline{toc}{subsubsection}{A cluster theory approach from
mutation invariants to Diophantine equations}

In this talk, we define and classify the sign-equivalent exchange
matrices. We give a Diophantine explanation for the differences between
rank 2 cluster algebras of finite type and infinite type. Then, we
classify the positive integer points of the Markov mutation invariant
and its variant. As an application, several classes of Diophantine
equations with cluster algebraic structures are exhibited. This is a
joint work with Zixu Li.

\subsection*{Hiroaki Karuo (Gakushuin
University)}\label{hiroaki-karuo-gakushuin-university}
\addcontentsline{toc}{subsection}{Hiroaki Karuo (Gakushuin University)}

\subsubsection*{Azumaya representations of generalized skein
algebras}\label{azumaya-representations-of-generalized-skein-algebras}
\addcontentsline{toc}{subsubsection}{Azumaya representations of
generalized skein algebras}

Some generalizations of skein algebras relate to (quantum) cluster
algebras of surface type. To study their algebraic structures,
understanding their representations is basic and important. For
non-commutative algebras, the unicity theorem claims that we can access
to the representation theory using their centers. In this talk, I will
explain some properties of generalized skein algebras related to the
unicity theorem and share some future works.

\subsection*{Davide Dal Martello (Rikkyo
University)}\label{davide-dal-martello-rikkyo-university}
\addcontentsline{toc}{subsection}{Davide Dal Martello (Rikkyo
University)}

\subsubsection*{Painlevé VI, symmetries, and
clusters}\label{painlevuxe9-vi-symmetries-and-clusters}
\addcontentsline{toc}{subsubsection}{Painlevé VI, symmetries, and
clusters}

The sixth Painlevé equation (PVI) admits a native
\(\mathfrak{sl}_{2}(\mathbb{C})\)-Fuchsian representation. Taking
advantage of a higher Teichmüller coordinatization for the corresponding
monodromy group, we give Okamoto's symmetry of PVI a realization on the
representation space in the language of cluster
\(\mathcal{X}\)-mutations.The explicit mutation formula admits dual
characterizations in geometric terms of colored associaheda and
star-shaped fat graphs, expanding the cluster state of the art for PVI.

\section*{Organizers}\label{organizers}
\addcontentsline{toc}{section}{Organizers}

\begin{itemize}
\item
  Osamu Iyama
\item
  Yoshiyuki Kimura
\item
  Tomoki Nakanishi (chief)
\item
  Hironori Oya
\end{itemize}

\end{document}
